%中文摘要
\begin{abstract}
兰州理工大学坐落于甘肃省省会兰州市,是甘肃省人民政府、教育部、国家国防科技工业局共建高校,甘肃省首批高水平大学建设高校。中西部高校基础能力建设工程、国家大学生创新型实验计划、教育部卓越工程师计划入选高校,国家国防教育特色学校。

学校前身是1919年创立的甘肃省立工艺学校;1958年,在组建兰州工学院的基础上,将甘肃交通大学并入,定名为甘肃工业大学;1965年,学校划归第一机械工业部,同时将东北重型机械学院和北京机械学院的水力机械、化工机械、石油矿场机械和焊接工艺及设备专业成建制全部迁入,并从湖南大学、合肥工业大学抽调一批教师来校工作;1998年,转制为“中央与地方共建,以地方管理为主”的院校;2003年,正式更名为兰州理工大学。经过百年的建设与发展,学校逐步形成了以“艰苦奋斗、自强不息、求真务实、开拓创新”为主要内涵的“红柳精神”,铸就了“奋进求是”的校训精神,基本建成了一流工科、坚实理科、特色文科,进入国内同类高校高水平大学行列。

学校现有19个学院、1个教学研究部,设有研究生院、温州研究生分院。有全日制在校生29975人,其中本科生22632人、研究生6398人、国际学生448人。有兰工坪校区、彭家坪校区两个校区,占地面积2430亩,校舍建筑面积121万平方米,图书馆馆藏图书216万册、电子图书122万册,实验室面积5万多平方米,教学科研仪器设备资产值4.6亿元。
\keywords{关键字1;关键字2;关键字n}
\end{abstract}

%英文摘要
\begin{enabstract}
Lanzhou University of Technology (LUT) is situated in Lanzhou, an important city on the ancient Silk Road and the capital city of Gansu Province, China. LUT was formerly called Gansu Provincial Technical College in 1919. Then it became Gansu University of Technology in 1958, and was renamed Lanzhou University of Technology in 2003. After nearly one hundred years of development, LUT has grown to be a top multi-disciplinary university, which features solid foundation in Engineering, increasing development in science fields, and unique characteristics in liberal arts.

LUT encompasses nine fields of study, i.e., Engineering, Science, Management, Literature, Legal Studies, Education, Medicine, Fine Arts and Economics. It holds 16 provincial key disciplines and 4 national defense disciplines, among which two disciplines—Engineering and Materials Science—rank among top 1% of ESI International Ranking. There are 5 postdoctoral research stations, 6 first-level discipline doctoral degree programs, 25 second-level discipline doctoral degree programs, 23 first-level discipline master degree programs, 92 second-level discipline master degree programs and 66 undergraduate programs in LUT, all of which are available for international students.

School of International Education (SIE) was founded in 2013, which is responsible for admission, management, Chinese language teaching as well as organization of culture exchange activities for international students. SIE has four divisions, i.e., Admission Office, Teaching Affairs Office, Student Affairs Office, and Chinese Language Teaching Center for International Students. LUT houses more than 400 international students from 34 countries.
\enkeywords{Keyword1; Keyword2; Keywordn}
\end{enabstract}